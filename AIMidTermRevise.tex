\documentclass[a4paper, 11pt]{article}

\author{Ossama Edbali}
\title{AI mid-term revision}

\begin{document}

\maketitle

\abstract{
In this article we are going to cover the topics learned during the first term of the module ``Intro to AI''.
}

\section{Intelligent agents and task environments}
We can think of an agent as an entity that takes a sequence of percepts through \textbf{sensors} from the environment and maps it to an action using \textbf{actuators} (it may affect the environment). The mapping is performed by an \textbf{agent function}.

A rational agent is one that tries to maximise its \textbf{performance measure} for each \textbf{percept sequence}. The performance measure evaluates any sequence of environment states.

Task environments --> PEAS concept and properties

There are several intelligent agents; some of them are:
\begin{itemize}
  \item Simple-reflex
  \item Model-based reflex agents --> handle parital observability and keep track of the percept sequence up to date (internal state)
  \item Goal-based agents
  \item Utility-based agents
  \item Knowledge-based agents
\end{itemize}

State representations --> atomic, factored (attr, value), structured (objects)

\end{document}