\documentclass[12pt]{article}

\usepackage{times}
\usepackage[T1]{fontenc}

\renewcommand{\familydefault}{\sfdefault}

%%% JS listing %%
\usepackage{listings}
\usepackage{color}
\definecolor{lightgray}{rgb}{.9,.9,.9}
\definecolor{darkgray}{rgb}{.4,.4,.4}
\definecolor{purple}{rgb}{0.65, 0.12, 0.82}

\lstdefinelanguage{JavaScript}{
  keywords={typeof, new, true, false, catch, function, return, null, catch, switch, var, if, in, while, do, else, case, break},
  keywordstyle=\color{blue}\bfseries,
  ndkeywords={class, export, boolean, throw, implements, import, this},
  ndkeywordstyle=\color{darkgray}\bfseries,
  identifierstyle=\color{black},
  sensitive=false,
  comment=[l]{//},
  morecomment=[s]{/*}{*/},
  commentstyle=\color{purple}\ttfamily,
  stringstyle=\color{red}\ttfamily,
  morestring=[b]',
  morestring=[b]"
}

\lstset{
   language=JavaScript,
   backgroundcolor=\color{lightgray},
   extendedchars=true,
   basicstyle=\footnotesize\ttfamily,
   showstringspaces=false,
   showspaces=false,
   numbers=left,
   numberstyle=\footnotesize,
   numbersep=9pt,
   tabsize=2,
   breaklines=true,
   showtabs=false,
   captionpos=b
}
%%% JS listing %%%

% Page settings

\topmargin 0.0cm
\oddsidemargin 0.2cm
\textwidth 16cm 
\textheight 21cm
\footskip 1.0cm


% Abstract

\newenvironment{mabstract}{%
\begin{quote} \bf}
{\end{quote}}

\title{Practical introduction to fold using JavaScript} 

\author
{Ossama Edbali\\
\\
\normalsize{School of Computer Science, University of Birmingham,}\\
\normalsize{Edgbaston, Birmingham, West Midlands B15 2TT}\\
\\
\normalsize{oxe410@student.bham.ac.uk}
}

\date{19/04/2015}

%%%%%%%%%%%%%%%%% END OF PREAMBLE %%%%%%%%%%%%%%%%


\begin{document} 

% Double-space the manuscript.

\baselineskip24pt

\maketitle 

% Abstract

\begin{mabstract}
	Fold (or reduce) is a higher-order consumer function that gives to the programmer
	a simple pattern of recursion for processing lists. It is an essential pattern
	in functional programming. It uses a combining function applied initially to a start
	value and an element of the list and processes recursively the items of the list
	building up a return value.
	In other words folding a list means to \textit{reduce} a list into a single value.
\end{mabstract}

\section*{Introduction}
This article is an introductory and practical tutorial on fold using JavaScript. Therefore theoretical
properties of fold as well as its universality are not covered here. For an insight on these topics
see \ldots
I am more focussed on the utility of this pattern in everyday programming as well as giving a taste
of a wider topic: functional programming.

\section*{Fold by examples}
Below you can find some functions implemented using the fold pattern. The first two are the actual
implementation of fold (right and left).

In these examples I will use some useful functions:

\textbf{Check if an object is an array}
\medskip
\begin{lstlisting}
function isArray (arr) {
	return Object.prototype.toString.call(arr) === '[object Array]';
}
\end{lstlisting}

\textbf{Array destructor - head and tail}
\medskip
\begin{lstlisting}
function dctor (arr) {
	return {
    	'hd': arr[0],
    	'tl': arr.slice(1)
    };
}
\end{lstlisting}

\subsection*{Fold right}
\medskip
\begin{lstlisting}
function foldR (arr, f, init) {
	if (isEmpty(arr)) {
		return init;
	}
	else {
		var p = dctor(arr);
		return f(foldR(p.tl, f, init), p.hd);
	}
}
\end{lstlisting}

\subsection*{Fold left}
\medskip
\begin{lstlisting}
function foldL (arr, f, init) {
	if (isEmpty(arr)) {
		return init;
	}
	else {
		var p = dctor(arr);
		return _foldL(p.tl, f, f(init, p.hd));
	}
}
\end{lstlisting}

\subsection*{Size}
Here we define an anonymous function with the following arguments:
\begin{itemize}
	\item \verb|a|: the accumulator value
	\item \verb|x|: the current value
\end{itemize}

\medskip
\begin{lstlisting}
function size (arr) {
	return foldL(arr, function (a, x) { return a + 1; }, 0);	
}
\end{lstlisting}

\subsection*{Sum}
\medskip
\begin{lstlisting}
function sum (arr) {
	return foldL(arr, function (a, x) { return a + x; }, 0);	
}
\end{lstlisting}

\subsection*{Reverse}
Here the accumulator is an array. We perform a fold from the right appending the current value
to the accumulator.

\medskip
\begin{lstlisting}
function reverse (arr) {
	return foldR(arr, function (a, x) {
		a.push(x);
		return a;
	}, []);
}
\end{lstlisting}


\subsection*{Filter}


\subsection*{Any}

\subsection*{Average}

\subsection*{Partition}

\subsection*{For each}

\subsection*{Map}

\subsection*{Composition}

\subsection*{Count}

\subsection*{Drop}

\subsection*{Flatten}

\subsection*{Insertion sort}

\subsection*{Join}

\subsection*{Remove consecutive}

\section*{Comments}


\section*{Contributing}


% \cite{}

\begin{thebibliography}{9}
	\bibitem{lamport94}
  	Leslie Lamport,
	\emph{\LaTeX: a document preparation system},
	Addison Wesley, Massachusetts,
  	2nd edition,
  	1994.
\end{thebibliography}

\end{document}

% References:
% http://www2.lib.uchicago.edu/keith/ocaml-class/pattern-matching.html