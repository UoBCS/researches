\documentclass[12pt]{article}

\usepackage{times}

% The following parameters seem to provide a reasonable page setup.

\topmargin 0.0cm
\oddsidemargin 0.2cm
\textwidth 16cm 
\textheight 21cm
\footskip 1.0cm


%The next command sets up an environment for the abstract to your paper.

\newenvironment{mabstract}{%
\begin{quote} \bf}
{\end{quote}}

\title{Practical introduction to fold using JavaScript} 

\author
{Ossama Edbali\\
\\
\normalsize{School of Computer Science, University of Birmingham,}\\
\normalsize{Edgbaston, Birmingham, West Midlands B15 2TT}\\
\\
\normalsize{oxe410@student.bham.ac.uk}
}

\date{19/04/2015}

%%%%%%%%%%%%%%%%% END OF PREAMBLE %%%%%%%%%%%%%%%%


\begin{document} 

% Double-space the manuscript.

\baselineskip24pt

\maketitle 

% Abstract

\begin{mabstract}
	Fold (or reduce) is a higher-order consumer function that gives to the programmer
	a simple pattern of recursion for processing lists. It is an essential pattern
	in functional programming. It uses a combining function applied initially to a start
	value and an element of the list and processes recursively the items of the list
	building up a return value.
	In other words folding a list means to \textit{reduce} a list into a single value.
\end{mabstract}

\section*{Introduction}
This article is an introductory and practical tutorial on fold using JavaScript. Therefore theoretical
properties of fold as well as its universality are not covered here. For an insight on these topics
see \ldots
I am more focussed on the utility of this pattern in everyday programming as well as giving a taste
of a wider topic: functional programming.

\section*{Fold by examples}
Below you can find some functions implemented using the fold pattern. The first two are the actual
implementation of fold (right and left).

\subsection*{Fold right}

\subsection*{Fold left}

\subsection*{Size}

\subsection*{Sum}

\subsection*{Reverse}

\subsection*{Filter}

\subsection*{Any}

\subsection*{Average}

\subsection*{Partition}

\subsection*{For each}

\subsection*{Map}

\subsection*{Composition}

\subsection*{Count}

\subsection*{Drop}

\subsection*{Flatten}

\subsection*{Insertion sort}

\subsection*{Join}

\subsection*{Remove consecutive}


\section*{Comments}


\section*{Contributing}


\section*{Formatting Citations}

Citations can be handled in one of three ways.  The most
straightforward (albeit labor-intensive) would be to hardwire your
citations into your \LaTeX\ source, as you would if you were using an
ordinary word processor.  Thus, your code might look something like
this:


\begin{quote}
\begin{verbatim}
However, this record of the solar nebula may have been
partly erased by the complex history of the meteorite
parent bodies, which includes collision-induced shock,
thermal metamorphism, and aqueous alteration
({\it 1, 2, 5--7\/}).
\end{verbatim}
\end{quote}


\noindent Compiled, the last two lines of the code above, of course, would give notecalls in {\it Science\/} style:

\begin{quote}
\ldots thermal metamorphism, and aqueous alteration ({\it 1, 2, 5--7\/}).
\end{quote}

Under the same logic, the author could set up his or her reference list as a simple enumeration,

\begin{quote}
\begin{verbatim}
{\bf References and Notes}

\begin{enumerate}
\item G. Gamow, {\it The Constitution of Atomic Nuclei
and Radioactivity\/} (Oxford Univ. Press, New York, 1931).
\item W. Heisenberg and W. Pauli, {\it Zeitschr.\ f.\ 
Physik\/} {\bf 56}, 1 (1929).
\end{enumerate}
\end{verbatim}
\end{quote}

\noindent yielding

\begin{quote}
{\bf References and Notes}

\begin{enumerate}
\item G. Gamow, {\it The Constitution of Atomic Nuclei and
Radioactivity\/} (Oxford Univ. Press, New York, 1931).
\item W. Heisenberg and W. Pauli, {\it Zeitschr.\ f.\ Physik} {\bf 56},
1 (1929).
\end{enumerate}
\end{quote}

That's not a solution that's likely to appeal to everyone, however ---
especially not to users of B{\small{IB}}\TeX\ \cite{inclme}.  If you
are a B{\small{IB}}\TeX\ user, we suggest that you use the
\texttt{Science.bst} bibliography style file and the
\texttt{scicite.sty} package, both of which we are downloadable from our author help site. You can also
generate your reference lists by using the list environment
\texttt{\{thebibliography\}} at the end of your source document; here
again, you may find the \texttt{scicite.sty} file useful.

Whether you use B{\small{IB}}\TeX\ or \texttt{\{thebibliography\}}, be
very careful about how you set up your in-text reference calls and
notecalls.  In particular, observe the following requirements:

\begin{enumerate}
\item Please follow the style for references outlined at our author
  help site and embodied in recent issues of {\it Science}.  Each
  citation number should refer to a single reference; please do not
  concatenate several references under a single number.
\item Please cite your references and notes in text {\it only\/} using
  the standard \LaTeX\ \verb+\cite+ command, not another command
  driven by outside macros.
\item Please separate multiple citations within a single \verb+\cite+
  command using commas only; there should be {\it no space\/}
  between reference keynames.  That is, if you are citing two
  papers whose bibliography keys are \texttt{keyname1} and
  \texttt{keyname2}, the in-text cite should read
  \verb+\cite{keyname1,keyname2}+, {\it not\/}
  \verb+\cite{keyname1, keyname2}+.
\end{enumerate}

\noindent Failure to follow these guidelines could lead
to the omission of the references in an accepted paper when the source
file is translated to Word97 via HTML.

\section*{Handling Math, Tables, and Figures}

Following are a few things to keep in mind in coding equations,
tables, and figures for submission to {\it Science}.

\paragraph*{In-line math.}  The utility that we use for converting
from \LaTeX\ to HTML handles in-line math relatively well.  It is best
to avoid using built-up fractions in in-line equations, and going for
the more boring ``slash'' presentation whenever possible --- that is,
for \verb+$a/b$+ (which comes out as $a/b$) rather than
\verb+$\frac{a}{b}$+ (which compiles as $\frac{a}{b}$).  Likewise,
HTML isn't tooled to handle certain overaccented special characters
in-line; for $\hat{\alpha}$ (coded \verb+$\hat{\alpha}$+), for
example, the HTML translation code will return [\^{}$(\alpha)$].
Don't drive yourself crazy --- but if it's possible to avoid such
constructs, please do so.  Please do not code arrays or matrices as
in-line math; display them instead.  And please keep your coding as
\TeX-y as possible --- avoid using specialized math macro packages
like \texttt{amstex.sty}.

\paragraph*{Displayed math.} Our HTML converter sets up \TeX\
displayed equations using nested HTML tables.  That works well for an
HTML presentation, but Word97 chokes when it comes across a nested
table in an HTML file.  We surmount that problem by simply cutting the
displayed equations out of the HTML before it's imported into Word97,
and then replacing them in the Word document using either images or
equations generated by a Word equation editor.  Strictly speaking,
this procedure doesn't bear on how you should prepare your manuscript
--- although, for reasons best consigned to a note \cite{nattex}, we'd
prefer that you use native \TeX\ commands within displayed-math
environments, rather than \LaTeX\ sub-environments.

\paragraph*{Tables.}  The HTML converter that we use seems to handle
reasonably well simple tables generated using the \LaTeX\
\texttt{\{tabular\}} environment.  For very complicated tables, you
may want to consider generating them in a word processing program and
including them as a separate file.

\paragraph*{Figures.}  Figure callouts within the text should not be
in the form of \LaTeX\ references, but should simply be typed in ---
that is, \verb+(Fig. 1)+ rather than \verb+\ref{fig1}+.  For the
figures themselves, treatment can differ depending on whether the
manuscript is an initial submission or a final revision for acceptance
and publication.  For an initial submission and review copy, you can
use the \LaTeX\ \verb+{figure}+ environment and the
\verb+\includegraphics+ command to include your PostScript figures at
the end of the compiled PostScript file.  For the final revision,
however, the \verb+{figure}+ environment should {\it not\/} be used;
instead, the figure captions themselves should be typed in as regular
text at the end of the source file (an example is included here), and
the figures should be uploaded separately according to the Art
Department's instructions.


\section*{What to Send In}

What you should send to {\it Science\/} will depend on the stage your manuscript is in:

\begin{itemize}
\item {\bf Important:} If you're sending in the initial submission of
  your manuscript (that is, the copy for evaluation and peer review),
  please send in {\it only\/} a PostScript or PDF version of the
  compiled file (including figures).  Please do not send in the \TeX\ 
  source, \texttt{.sty}, \texttt{.bbl}, or other associated files with
  your initial submission.  (For more information, please see the
  instructions at our Web submission site,
  http://www.submit2science.org/ .)
\item When the time comes for you to send in your revised final
  manuscript (i.e., after peer review), we require that you include
  all source files and generated files in your upload.  Thus, if the
  name of your main source document is \texttt{ltxfile.tex}, you
  need to include:
\begin{itemize}
\item \texttt{ltxfile.tex}.
\item \texttt{ltxfile.aux}, the auxilliary file generated by the
  compilation.
\item A PostScript file (compiled using \texttt{dvips} or some other
  driver) of the \texttt{.dvi} file generated from
  \texttt{ltxfile.tex}, or a PDF file distilled from that
  PostScript.  You do not need to include the actual \texttt{.dvi}
  file in your upload.
\item From B{\small{IB}}\TeX\ users, your bibliography (\texttt{.bib})
  file, {\it and\/} the generated file \texttt{ltxfile.bbl} created
  when you run B{\small{IB}}\TeX.
\item Any additional \texttt{.sty} and \texttt{.bst} files called by
  the source code (though, for reasons noted earlier, we {\it
    strongly\/} discourage the use of such files beyond those
  mentioned in this document).
\end{itemize}
\end{itemize}


% \cite{}

\begin{thebibliography}{9}
	\bibitem{lamport94}
  	Leslie Lamport,
	\emph{\LaTeX: a document preparation system},
	Addison Wesley, Massachusetts,
  	2nd edition,
  	1994.
\end{thebibliography}

\end{document}

% References:
% http://www2.lib.uchicago.edu/keith/ocaml-class/pattern-matching.html