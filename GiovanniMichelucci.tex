\documentclass[a4paper,11pt]{article}
\usepackage{hyperref}

\hypersetup{
    colorlinks,
    citecolor=black,
    filecolor=black,
    linkcolor=black,
    urlcolor=black
}

% Header %
\author{Ossama Edbali}
\title{Urban planning pioneer: Giovanni Michelucci}
\date{October 30, 2014}

\begin{document}

% Apply header %
\maketitle
\tableofcontents
\newpage

% Content %
\section{Abstract}
In this article we will look at an influential figure in Italian urban planning: Giovanni Michelucci.
We will analyse briefly his life and the major events that influenced his work.
The second part will deal with Michelucci's contributions to the Italian planning system, in particular
to Florence and Pistoia. We will look at some brilliant examples such as the planning of the Larderello Village.
In the last part of this article we are going to cover the impact of his work and thinking on contemporary planning.

\section{Context}
Before analysing Michelucci's life and work we must first understand the context in which he developed his thinking.

Official urban planning in Italy began with an Act called "Piano Regolatore" 1884 - Cesare Beruto which has planned the extension of Milan.

The figure of the \textbf{urban planner} had been recognised officially in the 30s with the \emph{Italian Rationalism} (in which Giovanni Michelucci was an active leader). During this period many cities and villages were built from scratch by the fascist regim. In fact, Mussolini wanted to ``restore the lost glorious Roman Empire'' by planning and building communities (the so-called \emph{planned communities}). During Mussolini's regime many architects, urban planners and engineers (including Michelucci) were influenced heavily by the regime's concepts; they could not express their thoughts in their own work.

In 1942 the governement issued the first general law of regional planning. The post-war period is the most productive in terms of house building and village planning. As a consequence there was a need to protect historical centres (Florence, Rome, Verona, Milano, Torino etc...) and green spaces. During the 50s and 60s there was a huge activist movement in order to tackle these issues.

This was the context in the first-half of the 20st century in which Michelucci contributed with his innovative ideas.

\section{Biography}
Giovanni Michelucci was an Italian urban planner and architect which has influenced the modern urban planning in Italy. In fact he is also referred as ``the patriarch of modern Italian urban planning''. He was born in Pistoia (near Florence, in Tuscany) in 1891 and died in 1990 two days before his hundredth birthday. Michelucci lived an entire century, assisted to the World Wars and as a consequence he had a real respect to human dignity.

In 1916 he was enrolled in the Italian Army in Caporetto during the First World War. During the war he planned his first architectural work: a small chapel for soldiers. The ideas behind the planning were very simple but it can be noticed his early talent in planning and architecture.

In 1920 he started the planning of the outskirts of Pistoia (Regio Istituto Nazionale d'Istruzione Professionale, Rome).
In 1933 Michelucci was a member of the \textbf{Gruppo Toscano} (Nello Baroni, Pier Niccolò Berardi, Italo Gamberini, Sarre Guarnieri, Leonardo Lusanna) which task was to plan the \textbf{Santa Maria Novella Railway Station}. The project immediately became famous because of the combination of modern and old as well as the planning rationalism.

Just before the Second World War he started to teach at the University of Florence and in 1945 was elected Head of the school of Architecture (at the time there wasn't a clearly defined and specific figure of urban planner). During the post-war period he started putting some ideas and hypothesis on how to rebuild the Ponte Vecchio neighbourhoods.

During the 50s Michelucci contributed to the planning of many worship areas such as Chiesa della Vergine a Pistoia, Chiesa della Beata Maria Vergine di Pomarance, Cappella di Lagoni di Sasso.

In 1961 Michelucci was working on the church of St. John the Baptist (nicknamed the Highway Church to remember those who had built the nearby highway). With no limitations of time, size, form or budget, he was inspired by Le Corbusier's Notre-Dame-du-Haut in Ronchamp in France.

During the 80s he organised cultural activities on issues such as housing conditions, suburbs, immigration, health (related to poor living conditions). In his last years, he participated with great pas�sion in the fundamental issues of the de�bate about the city.

\section{Philosophy, contributions and achievements}
In this section we will touch different aspects of Michelucci's thinking on urban planning such as public spaces,
social buildings and so forth. In the second part I am going to introduce you his most important achievements and contributions to urban planning.

\subsection*{Giovanni Michelucci's thinking on urban planning}
Before listing the most important contributions and achievements we are going to analyse Michelucci's philosophy on urban planning.
The key factors of urban planning are:
\begin{itemize}
  \item Particular attention to public space
  \item Outdoor rooms
  \item Public realm
  \item Connection of separate spaces in order to make a new ``drawing'' of the city
  \item Buildings must offer meeting places for people
  \item Concept of ``continuous building'': buildings should confluire in a public place
  \item Relationship between space and society
  \item Organic architecture
  \item Importance of the environment: built and natural environment are a unique entity
  \item Improve life quality by developing new techniques in urban planning such as public gardens in highly dense habitative places
  \item Particular attention to the working and poor classes
  \item Concept of the social city (like Garden cities)
\end{itemize}

During his career as an architect and urban planner, Michelucci often demolished his own work in order to adapt it to the development
of new techniques in planning (e.g. Palazzo delle Poste of Via Verdi).

The main figures in Michelucci's urban planning were people and their dignity.
Moreover, in his work he tried to merge the built and natural environment as much as possible.
With this in mind his main areas of research and intervention were: migration, housing exclusion, suburbs, urban security and health.
His main aim was to make policy makers aware of the conditions of poor living conditions in the so-called working class ``dormitory areas'' in the outskirts of big towns and cities.

In the 60s Michelucci developed new concepts of urban planning such as: variable city, refuse of regular planning (straight lines and standard patterns), relationship between old and modern planning, usage of new as well as traditional materials.

However his ideas were criticised by many politicians, architects and urban planners because of their \emph{radical} side.

\subsection*{Contributions and achievements}
In this section we are going to cover some of the most important works of Michelucci. He contributed mainly in the post-war planning of Florence which
has been place of many fierce shellings. Michelucci was involved also in the planning of small residential areas such as the outskirts of Pistoia and the Larderello village. All his projects had in common the concepts of open space and the importance of keeping the natural environment intact as much as possible.

Below you can find the most important Michelucci's plannings.

\subsubsection*{Santa Maria Novella Railway Station}
The project was set up by the "Gruppo Toscano" (Tuscan Group) in the 30s and it is the fourth largest Italian station. Michelucci and his colleagues had in mind a very important point: it was not just a matter of planning the station itself but they had to analyse the neighbourhood and the actual street network. In Italy it is regarded as one of the most important masterpieces of the Italian Rationalism.

A lot of criticism (mainly from traditionalists) arised because of the nearby Santa Maria Novella Church, which dominates the landscape with its campanile. However the planning group explained that the Church will remain dominant over the Piazza because its surrounding buildings (the station and the School of Carabinieri) have a flat and horizontal layout.

\subsubsection*{Highway Church}
The second landmark, the revolutionary tent-shaped church of St. John the Baptist was commissioned when Michelucci, a non-practising Christian, was 73 years old. The highway authority gave him a free reign in building this memorial to those who had died constructing the road.

This project can be seen more as a job of an architect rather than an urban planner but Michelucci considered the planning of this church as a planning of a small village surrounded by a green belt near a good transportation system.

In his words,
\begin{quote}
``This church is a little city in which men should meet and recognize in each other the common hope of finding each other again''
\end{quote}

\subsubsection*{Planning of Larderello Village: a good example of spatial organisation}
The Larderello Village (built from scratch) is one of the most impressive and successful urban planning story.
In 1954 the ``Larderello'' company (which at the time was managing the zona boracifera) charged Giovanni Michelucci with the task of implementing an urban plan of the industrial area of Larderello. At the time the company was growing very fast and wanted to extend the area for civil and social purposes (e.g. housing for workers and their families).

The planning of such an industrial area had inevitable difficulties as Michelucci stated:
``The plan was very industrious and many times I had to modify and adapt it.''

Nevertheless the Larderello village is the most successful expression of the Michelucci's philosophy on urban planning which is still modern nowadays. Concepts such as open space are natural and built environment as a unique entity were very important for the success of the planning.
The centre of

Michelucci had a group of assistants which developed the project of the residential are as well as the civil and social facilities. The three main buildings in the village are:
\begin{itemize}
  \item the Church, planned and built during in 1959;
  \item a new refinery for boric acid completed in 1957;
  \item the residential area for the managers built in 1956;
\end{itemize}

The residential area is strongly connected (recall the concept of connecting separate spaces) and surronded by a green belt. A very good and structured street network was provided in order to allow easy and safe travels to and from work. The village had not changed the natural configuration which is merged with the built environment. Despite the nearby factories, one does not perceive their presence in the countless small streets in the village.
The buildings in the residential area adhere to the characteristics of the soil (mostly mountainous) without having to transform it. 

Another aspect in the planning of the Larderello village was the use of local materials such as the white stone and brick.

\section{Impact on contemporary planning}
Michelucci's philosophy of urban planning is still alive today in various forms.
A clear example is the \textbf{Giovanni Michelucci Foundation} founded by himself back in the 1982, in which he bequeathed his vision and values, that he wanted to be attentive to the social problems of the city and to the separate worlds of total institutions as prison, asylum, hospitals. He committed the Foundation to offering ideas and plans for action on the chronic urban question, how to reconnect separate spaces by a new design of the city.

The Michelucci Foundation coordinates numerous projects and research in partnership with local authorities and cultural institutions, developing programs and proposals to integrate the local policies on the most relevant urban problems: migration, housing exclusion, suburbs, urban security, health.

The Foundation is very active today and organises conferences along with the local council on urban planning and architecture. They also carry on some incompleted projects that are still modern despite the time in which they have been set up. In other words the foundation can be regarded as sort of continuation of Michelucci's thinking on Italian contemporary urban planning.

Another example of impact on contemporary planning is the magazine of the Foundation, "La Nuova Citta" which was created in December 1945 (before the Foundation's birth). Since then, a regular series has followed with the intent to document and discuss \textbf{planning issues} with regards to the present city and the emerging social problems. The Foundation continues to publish the magazine today with specific publications of current activities, catalog of finished projects, and the multimedia production of materials.

\section{Conclusion and observations}
Through this article we have explored a very small part of this great Italian planning pioneer and architect. Nevertheless we can still conclude that Michelucci was not only an attentive observer of nature, of the space around him and of life but he also had an intimate understanding of the materials he used and where they came from. He saw the city as a \textbf{place for socialisation} where ``houses, neighbourhoods, zones were not for inhabitants, but for people''.

One of the crucial aspect of all his projects is that it does not exist incompatibility between past and present. It is the tradition that turns, takes on new aspects; the old and the modern must coexist in such a way that we can start to look at them as a unique entity.

Many of Michelucci's projects included social housing, workers' villages (e.g. Larderello), public buildings (e.g. SMN Railway Station), banks, churches, museums, social spaces for prisons, hospitals and schools. All of his projects were developed with the idea of a friendly, supportive, democratic community-city, where urban planning is open to the city and people-oriented (the human at the centre).

\newpage

\begin{thebibliography}{widest entry}
  \bibitem{cite_key1} Cresti, C. (1990) Scritti per Giovanni Michelucci. Firenze: Angelo Pontecorboli Editore.
  \bibitem{cite_key2} Giovanni Michelucci Foundation http://www.michelucci.it
  \bibitem{cite_key3} Trecada, E. (2011) Reformation Places and the Use of Senses in Their Design.
  \bibitem{cite_key4} La Nuova Citta' n2/IX, 2014 \url{http://www.michelucci.it/node/300}
  \bibitem{cite_key5} Firenze: la chiesa dell'Autostrada \url{http://www.youtube.com/watch?v=AsJj9G4a8ac}
  \bibitem{cite_key6} Giovanni Michelucci: gli edifici pubblici, interview http://www.scuola.rai.it/articoli/giovanni-michelucci-gli-edifici-pubblici/7465/default.aspx
  \bibitem{cite_key7} Marcetti C., Solimano N., Musumeci N., Colombo M., Bonfiglioli F. (2006) Il villaggio di Larderello. Rapporto sulle architetture e l'organizzazione spaziale e ambientale del territorio della geotermia
  \bibitem{cite_key8} C. Conforti, R. Dulio, M. Marandola, Giovanni Michelucci 1891-1990, Electa, Milano 2006.
  \bibitem{cite_key9} Bechi, P. (2003) Giovanni Michelucci borsa merci, in "Area" no 71.
\end{thebibliography}

\end{document}